 \documentclass{beamer}
\usetheme{Berlin}
\usecolortheme{beaver}
\usepackage[ngerman]{babel}
\usepackage{graphicx}
\usepackage[utf8]{inputenc}
\usepackage{times}
\usepackage[T1]{fontenc}
\usepackage{subfigure}
\usepackage{moreverb}

\title{Project R5: Eclipse view for task executions}
\author[]{Samy Dafir\\Dominik Baumgartner\\ Sophie Reischl }
\date{\today}
\begin{document}
\frame{\titlepage}

\begin{frame}
    \frametitle{Content} 
    \tableofcontents 
\end{frame}

\section{Aufgabenstellung}
\begin{frame}
	\frametitle{Aufgabenstellung}
    \begin{block}{Aufgabe:}
	    \begin{itemize}
		    \item Eclipse Plugin
		    \item Daten aus Files einlesen, diese dann grafisch darstellen 
            \item Mittels Nebula XY Graph            
	    \end{itemize}
    \end{block}
\end{frame}
\begin{frame}
	\frametitle{Data Files}
    \begin{block}{Binary File:}
	    \begin{itemize}
		    \item Data von Typ:\\
			\text{typedef struct monRec \{ }\\
			\text{\qquad double timeStamp;}\\
			\text{\qquad double value;}\\
			\text{\qquad double ID;}\\
			\text{\} MON\_RECORD;}\\
		    \item Jedes file: $Tasks\_<name\_of\_core>.vdt$
	    \end{itemize}
    \end{block}
\end{frame}

\begin{frame}
	\frametitle{Data Files}
	\begin{block}{HTML File:}
			\text{<actor name="'Core\_A1"'  type="'CPUSCHEDULER"'' ID="'2"'>}\\
			\text{\quad <itemlist type="'Task"' nrOfItems=“2"'>}\\
			\text{\quad\quad <item name="'Task\_LET\_DRV\_TASK\_A1"'>}\\
			\text{\quad\quad\quad <itemAttribute name="'ID"' value="'1"' />}\\
			\text{\quad\quad\quad <property name="'Priority"' value="'10"' />}\\
			\text{\quad\quad </item>}\\
			\text{\quad\quad <item name="'Task\_A1\_10msT1\_LET01"'>}\\
			\text{\quad\quad\quad <itemAttribute name="'ID"' value="'2"' />}\\
			\text{\quad\quad\quad <property name="'Priority"' value="'5"' />}\\
			\text{\quad\quad </item>}\\
			\text{\quad </itemlist>}\\	
			\text{</actor>}\\			
	\end{block}
\end{frame}

\begin{frame}
	\frametitle{User Eingaben:}
	\begin{block}{HTML File:}
		\begin{itemize}
			\item Benutzer wählt binary files und html files
			\item Danach werden die gewünschten Tasks ausgewählt
			\item Graph mit gewählten Tasks wird erstellt
		\end{itemize}	
	\end{block}
\end{frame}

\begin{frame}
	\frametitle{Display Output:}
	\begin{block}{Graph:}
		\begin{itemize}
			\item Graph in Cores eingeteilt
			\item Jeder Task nach Priorität in Cores eingeteilt
			\item Benutzer kann zoomen auf der Zeitachse
		\end{itemize}	
	\end{block}
\end{frame}
\section{Implementierung}
\begin{frame}
	\frametitle{Implementierung:}
	\begin{block}{}
	\end{block}
\end{frame}
\begin{frame}
\section{Beispiel}
	\frametitle{Beispiel:}
\begin{block}{}
\end{block}
\end{frame}
\end{document}

 \documentclass{beamer}
\usetheme{Berlin}
\usecolortheme{beaver}
\usepackage[ngerman]{babel}
\usepackage{graphicx}
\usepackage[utf8]{inputenc}
\usepackage{times}
\usepackage[T1]{fontenc}
\usepackage{subfigure}
\usepackage{moreverb}

\title{Project R5: Eclipse view for task executions}
\author[]{Samy Dafir\\Dominik Baumgartner\\ Sophie Reischl }
\date{\today}
\begin{document}
\frame{\titlepage}

\begin{frame}
    \frametitle{Content} 
    \tableofcontents 
\end{frame}

\section{Task - Overview}
\begin{frame}
	\frametitle{Overview}
    \begin{block}{What?}
	    \begin{itemize}
		    \item Eclipse Plugin
		    \item Read data out of files and display them graphically 
            \item Using Nebula XY Graph            
	    \end{itemize}
    \end{block}
\end{frame}
\begin{frame}
	\frametitle{Data Files}
    \begin{block}{Binary File:}
	    \begin{itemize}
		    \item Data von Typ:\\
			\text{typedef struct monRec \{ }\\
			\text{\qquad double timeStamp;}\\
			\text{\qquad double value;}\\
			\text{\qquad double ID;}\\
			\text{\} MON\_RECORD;}\\
		    \item Each file: $Tasks\_<name\_of\_core>.vdt$
	    \end{itemize}
    \end{block}
\end{frame}

\begin{frame}
	\frametitle{Data Files}
	\begin{block}{XML File:}
			\text{<actor name="'Core\_A1"'  type="'CPUSCHEDULER"'' ID="'2"'>}\\
			\text{\quad <itemlist type="'Task"' nrOfItems=“2"'>}\\
			\text{\quad\quad <item name="'Task\_LET\_DRV\_TASK\_A1"'>}\\
			\text{\quad\quad\quad <itemAttribute name="'ID"' value="'1"' />}\\
			\text{\quad\quad\quad <property name="'Priority"' value="'10"' />}\\
			\text{\quad\quad </item>}\\
			\text{\quad\quad <item name="'Task\_A1\_10msT1\_LET01"'>}\\
			\text{\quad\quad\quad <itemAttribute name="'ID"' value="'2"' />}\\
			\text{\quad\quad\quad <property name="'Priority"' value="'5"' />}\\
			\text{\quad\quad </item>}\\
			\text{\quad </itemlist>}\\	
			\text{</actor>}\\			
	\end{block}
\end{frame}

\begin{frame}
	\frametitle{User Input:}
	\begin{block}{XML File:}
		\begin{itemize}
			\item User selects binary files und xml files
			\item Then task list displayed
			\item Graph with chosen tasks created
		\end{itemize}	
	\end{block}
\end{frame}

\begin{frame}
	\frametitle{Display Output:}
	\begin{block}{Graph:}
		\begin{itemize}
			\item Graph divided into each core
			\item Each task is displayed in its own horizontal band
			\item Tasks are ordered by priority
			\item User can zoom in and choose time intervals
		\end{itemize}	
	\end{block}
\end{frame}

\section{Implementation}
\begin{frame}
	\frametitle{Implementation}
	\begin{block}{What do we start with?}
		xml file: task name, id, priority
		binary files: id, states, tiestamps
	\end{block}
\end{frame}

\begin{frame}
	\begin{block}{What is there to do?}
		\begin{enumerate}
			\item Parse xml file
			\item Select processes from list
			\item Parse binary files $\rightarrow$ extract state info
			\item Map process to all its states
			\item Insert all processes into xy Graph
		\end{enumerate} 
	\end{block}
\end{frame}

\begin{frame}
	\frametitle{Parse XML}
	\begin{itemize}
		\item Parse xml with simple DOM parser.\\
		\item Create HashMap of all processes.\\ 
	\end{itemize}
	
	\begin{tabular}{|c|c|}
		\hline 
		\textbf{Taskname} & \textbf{TaskInfo} \\ 
		\hline 
		Taskname 1 & id = 4, priority = 8 \\ 
		\hline 
		Taskname 2 & id = 2, priority = 4 \\ 
		\hline 
		Taskname 3 & id = 3, priority = 9  \\ 
		\hline 
		...    & ...  \\ 
		\hline 
	\end{tabular} 
\end{frame}

\begin{frame}
	\frametitle{Parse binary files}
	Only get relevant info: Processes the user selected
	\includegraphics[width = 7cm]{table.png}
\end{frame}

\begin{frame}
	\frametitle{Parse binary files}
	\begin{block}{}
		\begin{itemize}
			\item Get selected ids from HashMap 
			\item Go through complete binary file
			\item Read state and timestamp info
			\item Only record states if ID selected
			\item All states for ID collected in HashMap
		\end{itemize}
	\end{block}
\end{frame}

\begin{frame}
	\frametitle{Parse binary files}
	\textbf{Resulting HashMap:}\\
	\vspace{1cm}
\begin{tabular}{|c|c|}
		\hline 
		\textbf{ID} & \textbf{StateInfo}  \\
		\hline 
		1 & List(states, timestamps) \\
		\hline 
		2 & List(states, timestamps)  \\ 
		\hline 
		... & ...  \\ 
		\hline 
	\end{tabular}
\end{frame}

\begin{frame}
	\frametitle{Combine process and state info}
	\begin{block}{Combine State and TaskInfo}
		\begin{itemize}
			\item contains all relevant info
			\item TreeMap provides instrinsic sorting
			\item Prefill set with TaskInfo
			\item For each entry: get States from HashMap
		\end{itemize}
	\end{block}
	\vspace{2mm}
	\begin{tabular}{|c|c|}
		\hline 
		\textbf{TraceInfo}  \\
		\hline 
		name1, core1, priority1, stateList1 \\
		\hline 
		name2, core2, priority2, stateList2  \\ 
		\hline 
		...  \\ 
		\hline 
	\end{tabular}
\end{frame}

\begin{frame}
	\frametitle{Build graph}
	\begin{block}{}
		\begin{itemize}
			\item Traces sorted by core and priority
			\item Traces contain states (1-4)
			\item Iterate over tree
			\item Calculate offset for each task
			\item Add to graph
		\end{itemize}
	\end{block}
\end{frame}




\begin{frame}
\section{Example}
	\frametitle{Example:}

\end{frame}
\end{document}
